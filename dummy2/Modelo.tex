		\subsection{Descripción del Entorno Biológico.}
			\normalsize{Consid\'erese un organismo con acceso a dos recursos para su supervivencia. Sean $S_1$ y $S_2$ las concentraciones de estos dos recursos contenidos en un chemostat. Entonces el vector:}
        \begin{equation}
        I=\binom{S_{1}}{S_{2}},
        \end{equation}

        \normalsize{constituye la condición ambiental para el consumidor.\citep{Diekmann2001, diekman2003}.}\\

        \normalsize{Ahora bien, estos organismos pueden especializarse de mayor o menor medida en el consumo de cualquiera de los 2 recursos a su disposici\'on. Describimos esto en un rasgo cuantizable $x$ que var\'ia continuamente entre $[0,1]$. Si $x$ equivale a 0, sólo el recurso $S_2$ es consumido, caso contrario cuando equivale 1, s\'olo se consume el recurso $S_1$. Extendiendo a escala poblacional, el impacto causado por el rasgo $x$ se toma impl\'icitamente dentro de los coeficientes $\eta(x)$ y $\xi(x)$, definidos de tal manera que la proporci\'on de ingesta \textit{per c\'apita} de un solo organismo con rasgo $x$ equivale, respectivamente a: $\eta(x)$ $S_1$ y $\xi(x)$ $S_2$.\citep{dieckman2005}}\\

        \normalsize{En el caso de una población monomórfica, la dinámica ecológica está gobernada por el siguiente sistema de ecuaciones diferenciales:}
        \begin{equation}\label{eqmono}
        \begin{split}
             \frac{d S_1}{dx}&=S_{01}-S_1-\eta(x)S_1X\\  \frac{d S_2}{dx}&=S_{02}-S_2-\xi(x)S_2X\\ \frac{d X}{dx}&=-X+\eta(x)S_1X+\xi(x)S_2X 
        \end{split}
        \end{equation}

        \normalsize{Donde X representa la densidad de la población consumidora y $S_{01}$ es la concentración del recurso i ($i\in\{1,2\}$) en el medio de entrada, para este modelo, la tasa de crecimiento poblacional de los consumidores con el rasgo $x$ bajo condiciones ambientales estables $I$ ($\frac{d X}{dx}=0$ en la tercera expresi\'on de \eqref{eqmono}), está dada por:}
        \begin{equation}\label{r}
        	\begin{split}
        			0&=r(x,I)X\\
            r(x,I)&=-1+\eta(x)S_1+\xi(x)S_2.
        	\end{split}
        \end{equation}
        
	\normalsize{Por lo tanto la primera condici\'on para estados estables es:}
				\begin{equation*}
					r(x,I)=0
				\end{equation*}
				
\normalsize{Para las 2 primeras expresiones de \eqref{eqmono} obtenemos lo siguiente para las mismas condiciones de estabilidad:}
        \begin{equation*}
        \begin{split}
             0&=S_{01}-S_1-\eta(x)S_1X\\  0&=S_{02}-S_2-\xi(x)S_2X
        \end{split}
        \end{equation*}
				\normalsize{Con la que, obteniendo $S_1$ y $S_2$ en t\'erminos de $X$ y sutituyendo en nuestra primera condici\'on de estabilidad $r=0$ tenemos:}
				
				\begin{equation*}
					-1+\frac{\eta(x)S_{01}}{1+\eta(x)X}+\frac{\xi(x)S_{02}}{1+\xi(x)X}=0
				\end{equation*}
				
        \normalsize{Que es una funci\'on mon\'otona decreciente de $X$ con l\'imite -1 para $X\to\infty$, y que tiene soluci\'on positiva para X=0 solo si:}
        \begin{equation}\label{global}
            \eta(x)S_{01}X+\xi(x)S_{02}X>1,
        \end{equation}
			\normalsize{entonces \eqref{eqmono} tiene un \'unico estado estable no trivial que es asint\'otico globalmente, dado que cumpla \eqref{global}.}

		\subsection{Sistemas de Ecuaciones de Selección-Mutación y Paso al Límite para Mutaciones.}

        \normalsize{Si la reproducción no es completamente fiel (aparece alguna mutaci\'on), un consumidor con el rasgo $y$ puede generar descendencia con el rasgo $x$. Sea $K(x,y)$ la densidad de probabilidad correspondiente. Sea $n(t,.)$ la densidad de consumidores en el tiempo t. El sistema \citep{dieckman2005}:}

        \begin{equation}\label{core}
            \begin{split}
                \frac{d S_1}{dt}(t)&=S_{01}-S_1(t)-S_1(t)\int_{0}^{1}\eta(y)n(t,x)dx,\\
               \frac{d S_1}{dt}(t)&=S_{02}-S_2(t)-S_2(t)\int_{0}^{1}\xi(x)n(t,x)dx\\
              \frac{d n}{dt}(t,x)&=-n(t,x)+\int_{0}^{1}K(x,y)[S_1(t)\eta(y)+S_2(t)\xi(x)]n(t,y)dy,
            \end{split}
        \end{equation}

        \normalsize{describe la interacción, a través de los recursos, de los diversos tipos de consumidores, así como el efecto de la mutación, le denominaremos sistema de ecuaciones de selección-mutación. Tomamos que la descendencia de un individuo con el rasgo $x$ tiene una distribución de rasgos descrita por la densidad K(x,.).}\\

        \normalsize{Adem\'as, asumiendo que las mutaciones son muy peque\~nas, de tal manera que la distribuci\'on de probabilidad $K(x,y)$ es muy peque\~no para $x$ fuera de un vecindario de radio $\varepsilon$ con ''centro'' en $y$, $\varepsilon>0$ muy peque\~no, entonces tomamos $K(x,y)\to K_{\varepsilon}(x,y)$ dependiente de este pequeño parámetro $\varepsilon$.\citep{dieckman2005}}\\

        \normalsize{Reescalamos el tiempo sustituyendo $\tau=\varepsilon$t (este escalamiento ajusta la escala temporal de modo que, al hacer $\varepsilon$ desaparecer, la escala de tiempo se adapte para observar el efecto de las mutaciones), reescribimos \eqref{core}:}

        \begin{equation}\label{nocon}
            \frac{\varepsilon}{n(\tau,x)}\frac{d n(\tau,x)}{d\tau}=-1+\int_{0}^{1}K(x,y)_{\varepsilon}[S_1(\tau)\eta(y)+S_2(\tau)\xi(x)]\frac{n(y,\tau)}{n(\tau,x)}dy.
        \end{equation}

        \normalsize{a la que a la vez podemos luego realizar la siguiente transformación}

        \begin{equation*}
            \varphi(\tau,x)=\varepsilon ln[n(\tau,x)],
        \end{equation*}
        \normalsize{entonces:}
        \begin{equation*}
            \frac{d\varphi(\tau,x) }{d\tau}=\int_{0}^{1}K_{\varepsilon}(x,y)[S_1(\tau)\eta(y)+S_2(\tau)\xi(x)]e^{\frac{\varphi(\tau,y)-\varphi(\tau,x)}{\varepsilon}}dy
        \end{equation*}

        \normalsize{Aprovechando lo mencionado anteriormente sobre $K_{\varepsilon}(x,y)$ realizamos el cambio de variable de integración $y=x+\varepsilon z$ y de la definici\'on de derivada parcial:}

        \begin{equation*}
            \frac{\varphi(\tau,y)-\varphi(\tau,x)}{\varepsilon} \to \frac{d\varphi(\tau,x) }{dx}z
        \end{equation*}

        \normalsize{y asumimos que la probabilidad de aparición de un nuevo rasgo como resultado de una mutación depende únicamente de la distancia al rasgo original. Por lo tanto, reemplazamos el kernel $K_{\varepsilon}$ por un kernel de convolución $\widetilde{K}$:}
        
        \begin{equation*}
            K_{\varepsilon}(x,y)dy\longrightarrow \widetilde{K}(z)dz
        \end{equation*}

        \normalsize{Donde $\widetilde{K}$ es una función no negativa y par definida en $(-\infty,+\infty)$, cuya integral es igual a 1. Al tomar formalmente el límite cuando $\varepsilon\to 0$ en \eqref{nocon}, obtenemos:}\\

        \begin{equation}\label{hj}
            \frac{d\varphi(t,x) }{dx}=-r(x,I)+[S_1(t)\eta(y)+S_2(t)\xi(x)]\Ham(\frac{\partial \varphi(t,x) }{\partial x})
        \end{equation}
            
        \normalsize{la ecuaci\'on de Hamilton-Jacobi, d\'onde r se define en \eqref{r} y $\Ham$ se toma como:}
        \begin{equation*}
             \Ham(p)=\int_{-\infty}^{\infty}\widetilde{K}(z)e^{-pz}dz-1
        \end{equation*}

        \normalsize{y que cumple $\Ham(0)=0$ y que, para una función par $\widetilde{K}$, $\Ham'(0)>0$; por lo tanto, $\Ham$ es convexa. Llamamos a $\Ham$ el Hamiltoniano correspondiente a $\widetilde{K}$.}