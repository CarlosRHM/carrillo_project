\section{¿Qué es la Dinámica Adaptativa?}{

        \subsection{Adaptación y Evolución de los Sistemas Biológicos.}{



            \normalsize{En el estudio de los sistemas biológicos, los conceptos de adaptación y evolución son fundamentales para comprender cómo los organismos responden a su entorno y cómo estas respuestas moldean la diversidad de la vida en la Tierra. \citep{adaptacion} La adaptación se refiere al proceso mediante el cual los organismos desarrollan características que mejoran su capacidad para sobrevivir y reproducirse en condiciones ambientales específicas. Estas características pueden ser de naturaleza física, como la presencia de pelaje grueso en animales que habitan regiones frías; fisiológica, como la capacidad de las plantas desérticas para retener agua; o conductual, como la migración estacional de las aves. La adaptación resulta principalmente de la selección natural, un mecanismo que favorece aquellas variaciones genéticas que proporcionan ventajas competitivas en un entorno particular.}\\

            \normalsize{\citep {evolucion} Por otro lado, la evolución es un proceso a largo plazo que describe los cambios genéticos acumulativos en las poblaciones de organismos a lo largo de generaciones. Este fenómeno, impulsado por la selección natural, las mutaciones genéticas, la deriva genética y el flujo génico, es responsable de la aparición de nuevas especies y de la vasta diversidad de formas de vida observada hoy en día. La evolución permite entender cómo los organismos han cambiado con el tiempo para adaptarse a condiciones ambientales cambiantes, y cómo estos cambios han contribuido al desarrollo de estructuras, comportamientos y funciones altamente especializadas.}\\

            \normalsize{La relación entre adaptación y evolución es íntima y complementaria. Mientras que la adaptación representa la respuesta inmediata de un organismo o una población a presiones selectivas específicas, la evolución engloba los cambios genéticos a largo plazo que consolidan estas adaptaciones en las poblaciones.}\\

            \normalsize{En el contexto evolutivo, es importante entender los conceptos de genotipo y fenotipo, pues la interacción entre ellos permite que los organismos se adapten a condiciones cambiantes.}
            
            \normalsize{El genotipo de un organismo se refiere al conjunto específico de genes que posee, determinado por los cromosomas que hereda. Es la base genética que define las características potenciales de un organismo. En especies asexuales y sexuales haploides, el genotipo se representa por un cromosoma de cada tipo, mientras que en especies diploides está formado por pares de cromosomas homólogos.}\\

            \normalsize{Cualquier característica de un individuo que esté determinada (en alguna medida) por el genotipo se denomina fenotipo (o rasgo fenotípico) y, por ende, es una característica heredable de los padres a la descendencia. Casi cualquier tipo de característica imaginable depende de los genes, desde rasgos físicos, como el tamaño del cuerpo o los colores, hasta rasgos mentales, como el carácter, la disposición y la inteligencia. Los fenotipos pueden ser discretos, cuando presentan diferencias claras y definidas (por ejemplo el tipo de sangre), o continuos, cuando varían en un rango medible (como la altura o peso {\citep{Dercole}.}\\

            

            \normalsize{La variabilidad fenotípica dentro de las poblaciones, conocida como polimorfismo, es crucial para la evolución y la adaptación de los organismos. Los fenotipos discretos suelen depender de uno o pocos genes y tienen poca influencia del ambiente, mientras que los fenotipos continuos están influenciados tanto por múltiples genes como por el entorno. Los fenotipos que proporcionan ventajas adaptativas son seleccionados de manera natural, lo que, con el tiempo, lleva a cambios genéticos en la población. Así, la variabilidad genética y fenotípica es esencial para la supervivencia y la evolución de las especies frente a desafíos ambientales y presiones selectivas {\citep{Dercole}.}\\

            \normalsize{\citep{Mirrahimi} Desde la década de 1980, el término \textit{evolución adaptativa} se ha acuñado para describir los formalismos matemáticos que abordan la selección y evolución de una caracter\'iztica cuantitativa. Sin embargo en este trabajo se ocupa siguiente enfoque \citep{Dieckmann1996, metz96}: se estudia la evoluci\;on fenot\'ipica causada por mutaciones raras (en el sentido temporal) ignorando sexo y genes.}

%                \begin{itemize}
%                    \item {
%                    
%                        \normalsize{La multiplicación de la población.}
%
%                    }
%                    
%                    \item {
%                    
%                        \normalsize{La selección mediante competencia por los recursos disponibles.}
%                    }
%                    
%                    \item {
%                    
%                        \normalsize{Mutaciones.}
%
%                    }
%                \end{itemize}
                
            \normalsize{Modelos simples basados en estos principios pueden explicar como emergen rasgos mas aptos y, a su vez, como poblaciones caracterizadas por varios rasgos bien diferenciados pueden coexistir potencialmente. Las simulaciones numéricas pueden presentar la aparición de ciclos y la especiacion, esto debido a que los recursos limitados generan competencia; los individuos con características similares utilizan resucursos similares, dando asi, una competencia mayor entre ellos. La cuestión de comprender cómo, en un población de este tipo, una especie mutante puede invadir o no una población inicial. En modelos de población cerrada, las mutaciones forman parte de la dinámica y se toman en cuenta la heredad de los rasgos ligeramente diferente a los progenitores.}
        }
