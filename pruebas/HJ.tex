\subsection{Ecuación de Hamilton-Jacobi.}
Desarrollamos la idea principal detr\'as de la ecuaci\'on con al que trabajaremo://
Sea una ecuaci\'on parcial diferencial no lineal \citep{evans} de la forma:
\begin{equation}\label{og}
	F(x;u,\nabla u)=0
\end{equation}
con $x\in\Omega\subseteq\Reals^n$, si renombramos $p\doteq\nabla u$ notaci\'on vectorial, asumimos que $F=F(x,u,p)$ es una funci\'on continua $F:\Reals^n$x$\Reals$x$\Reals^n \to \Reals$.\\
Dado $u(x)=\overline{u}(x)$ con $x\in\partial\Omega$ se construye una soluci\'on (al menos localmente, en la vecindad de la frontera) por el m\'etodo de caracter\'izticas.
Fijando un punto $x\in\partial\Omega$ consid\'erese una curva parametrizada por $t:x(t)$ con $x(0)=y$, y con:
\begin{equation*}
	\begin{split}
		u(t)&\doteq=u(x(t))\\
		p(t)&\doteq p(x(s))=\nabla u(x(t))
	\end{split}
\end{equation*}
Denotando por un punto la derivada con respecto a $t$ tenemos:
\begin{equation}\label{casi}
	\begin{split}
		\dot{u}&=\sum_i \frac{\partial u}{\partial x_i}\dot x_i=\sum_i p_i \dot x_i\\
		\dot p_j&=\sum_i \frac{\partial^2 u}{\partial x_i\partial x_j} \dot x_i
	\end{split}
\end{equation}

Diferenciando \eqref{og} con respecto a $x_j$:

\begin{equation*}
	\begin{split}
		\frac{\partial F}{\partial x_j}+\frac{\partial F}{\partial u}\frac{\partial u}{\partial x_j}+\sum_i \frac{\partial F}{\partial p_i}\frac{\partial^2 u}{\partial x_i\partial x_j}&=0\\
		-\frac{\partial F}{\partial x_j}-\frac{\partial F}{\partial u}p_j&=\sum_i \frac{\partial F}{\partial p_i}\frac{\partial^2 u}{\partial x_j\partial x_i}
	\end{split}
\end{equation*}

Al usar \eqref{casi} renombramos $\dot x_i=\frac{\partial F}{\partial p_i}$ obtenemos entonces:
\begin{equation*}
	\begin{split}
			\dot x_i&=\frac{\partial F}{\partial p_i}\\
			\dot{u}&=\sum_i p_i\frac{\partial F}{\partial p_i}\\
			\dot p_j&=-\frac{\partial F}{\partial x_j}-\frac{\partial F}{\partial u}p_j
	\end{split}
\end{equation*}
Que ahora, para un tipo espec\'ifico de problema supongamos que $F$ no depende explicitamente de $u$, entonces recuperamos las ecuaciones can\'onicas de Hamilton y podemos escribir $F\to\Ham$, y en notaci\'on vectorial.

\begin{equation*}
	\begin{split}
			\dot x&=\frac{\partial \Ham}{\partial p}\\
			\dot{u}&=p\cdot \frac{\partial \Ham}{\partial p}\\
			\dot p&=-\frac{\partial \Ham}{\partial x}
	\end{split}
\end{equation*}
con $x(0)=y$, $u(0)=u(y)$ y $p(0)=\nabla u(y)$.
De esta forma, podemos aplicar el formalismo de Hamilton-Jacobi a una variedad de problemas no necesariamente mec\'anicos, solamente pidiendo que la ecuaci\'on original cumpla $F=F(t;x,\nabla u)$. 


    