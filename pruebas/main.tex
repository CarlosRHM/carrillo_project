\documentclass[12pt]{article}

\usepackage[utf8]{inputenc}
\usepackage[spanish]{babel}
\usepackage[dvipsnames]{xcolor}
\usepackage{amssymb}
\usepackage{physics}
\usepackage{amsmath}
\usepackage{anysize}
\usepackage{multicol} 
\usepackage{graphicx}                      
\usepackage{blindtext}                                          
\usepackage{cancel}
\usepackage{tikz}
\usepackage[square,numbers]{natbib}

\usepackage{hyperref}
 \hypersetup{ colorlinks = true, linkcolor = black } %==== Hipervínculos ====%
\usepackage{geometry}
\newgeometry{ bottom = 2.54cm, top = 2.54cm, left = 2.54cm, right = 2.54cm } %==== Modificación de Márgenes ====%

\usepackage{fancyhdr}  
\pagestyle{fancy}
\fancyhf{}
\fancyhead[R]{Mecánica Clásica}
\fancyhead[L]{\thepage}
\renewcommand{\headrulewidth}{0.08pt} %==== Encabezados ====%

\renewcommand{\footrulewidth}{0.08pt} %==== Pie de Página ====%
\fancyfoot[L]{}
\fancyfoot[R]{\rightmark}

%\usepackage[pages=all]{background} 
%\backgroundsetup{ scale=1, color=black, opacity=0.15, angle=0,
 %   contents={ \includegraphics{logo_ifuap.png} }
%}

\setcounter{tocdepth}{3}

\newcommand{\Title}[1]{\begin{center} \LARGE{\textbf{\textit{#1}}} \end{center}}
\newcommand{\Abstract}[1]{\begin{abstract} \normalsize{#1} \end{abstract}}
\newcommand{\Theorem}[1]{\begin{center} \normalsize{\textit{#1}} \end{center}}

\newcommand{\blue}  { \color{blue} }
\newcommand{\red}   { \color{red} }
\newcommand{\green} { \color{OliveGreen} }
\newcommand{\orange}{ \color{orange} }

\newcommand{\deq}{:=}
\newcommand{\less}{<}
\newcommand{\grow}{>}

\newcommand{\Identity}{\mathbb{I}}
\newcommand{\Reals}{\mathbb{R}}
\newcommand{\Naturals}{\mathbb{N}}
\newcommand{\Integers}{\mathbb{Z}}
\newcommand{\Complex}{\mathbb{C}}
\newcommand{\Imaginaries}{\mathbb{I}}

\newcommand{\Lag}{\mathcal{L}}
\newcommand{\Ham}{\mathcal{H}}

\newcommand{\cbk}[1]{ \left( #1 \right) }
\newcommand{\sbk}[1]{ \left[ #1 \right] }
\newcommand{\kbk}[1]{ \left\{ #1 \right\} }
\newcommand{\tbk}[1]{ \langle #1 \rangle }

\newcommand{\Grad}{ \vec{\nabla} }
\newcommand{\Divg}{ \vec{\nabla} \cdot }
\newcommand{\Curl}{ \vec{\nabla} \times }
\newcommand{\Lapl}{ \nabla^{2} }

\newcommand{\rad}[1]{ \vec{r}_{#1} }

\newcommand{\define}[3]{ \left. #1 \right|_{#2}^{#3} }
\newcommand{\Sum}[3]{ \sum_{#1=#2}^{#3} }
\newcommand{\Tensor}[3]{ #1^{#2}_{#3} }

\newcommand{\Partial}[2]{ \frac{\partial#1}{\partial#2} }
\newcommand{\SecPartial}[3]{ \frac{\partial^{2}#1}{\partial#2\partial#3} }

\newcommand{\eqLagrange}[1]{ \frac{d}{dt}\cbk{\Partial[\dot{#1}]{\Lag}} - \Partial[#1]{\Lag} = 0 }

\newcommand{\eqHamilton}[2]{ \Partial[#1_{i}]{\Ham} = - \dot{#2}_{i}, ~ ~ \Partial[#2_{i}]{\Ham} = \dot{#1}_{i} }

\begin{document}

	\Title{Evoluci\'on de un rasgo cuantitativo en una poblaci\'on monom\'orfica, enfoque con formalismo de Hamilton-Jacobi.}

	\Abstract{Reproducimos parte de los resultados expuestos en el articulo \citep{dieckman2005}, enfoc\'andonos en la evoluci\'on de un rasgo cuantitativo de una poblaci\'on monom\'orfica, y obteniendo resultados con un nuevo enfoque sobre las ecuaciones de selecci\'on-mutaci\'on utilizadas, en este caso con el formalismo de Hamilton-Jacobi.}
	\tableofcontents
	\clearpage
%-----------------------------------------------------------------
	\section{¿Qué es la Dinámica Adaptativa?}{

        \subsection{Adaptación y Evolución de los Sistemas Biológicos.}{



            \normalsize{En el estudio de los sistemas biológicos, los conceptos de adaptación y evolución son fundamentales para comprender cómo los organismos responden a su entorno y cómo estas respuestas moldean la diversidad de la vida en la Tierra. \citep{adaptacion} La adaptación se refiere al proceso mediante el cual los organismos desarrollan características que mejoran su capacidad para sobrevivir y reproducirse en condiciones ambientales específicas. Estas características pueden ser de naturaleza física, como la presencia de pelaje grueso en animales que habitan regiones frías; fisiológica, como la capacidad de las plantas desérticas para retener agua; o conductual, como la migración estacional de las aves. La adaptación resulta principalmente de la selección natural, un mecanismo que favorece aquellas variaciones genéticas que proporcionan ventajas competitivas en un entorno particular.}\\

            \normalsize{\citep {evolucion} Por otro lado, la evolución es un proceso a largo plazo que describe los cambios genéticos acumulativos en las poblaciones de organismos a lo largo de generaciones. Este fenómeno, impulsado por la selección natural, las mutaciones genéticas, la deriva genética y el flujo génico, es responsable de la aparición de nuevas especies y de la vasta diversidad de formas de vida observada hoy en día. La evolución permite entender cómo los organismos han cambiado con el tiempo para adaptarse a condiciones ambientales cambiantes, y cómo estos cambios han contribuido al desarrollo de estructuras, comportamientos y funciones altamente especializadas.}\\

            \normalsize{La relación entre adaptación y evolución es íntima y complementaria. Mientras que la adaptación representa la respuesta inmediata de un organismo o una población a presiones selectivas específicas, la evolución engloba los cambios genéticos a largo plazo que consolidan estas adaptaciones en las poblaciones.}\\

            \normalsize{En el contexto evolutivo, es importante entender los conceptos de genotipo y fenotipo, pues la interacción entre ellos permite que los organismos se adapten a condiciones cambiantes.}
            
            \normalsize{El genotipo de un organismo se refiere al conjunto específico de genes que posee, determinado por los cromosomas que hereda. Es la base genética que define las características potenciales de un organismo. En especies asexuales y sexuales haploides, el genotipo se representa por un cromosoma de cada tipo, mientras que en especies diploides está formado por pares de cromosomas homólogos.}\\

            \normalsize{Cualquier característica de un individuo que esté determinada (en alguna medida) por el genotipo se denomina fenotipo (o rasgo fenotípico) y, por ende, es una característica heredable de los padres a la descendencia. Casi cualquier tipo de característica imaginable depende de los genes, desde rasgos físicos, como el tamaño del cuerpo o los colores, hasta rasgos mentales, como el carácter, la disposición y la inteligencia. Los fenotipos pueden ser discretos, cuando presentan diferencias claras y definidas (por ejemplo el tipo de sangre), o continuos, cuando varían en un rango medible (como la altura o peso {\citep{Dercole}.}\\

            

            \normalsize{La variabilidad fenotípica dentro de las poblaciones, conocida como polimorfismo, es crucial para la evolución y la adaptación de los organismos. Los fenotipos discretos suelen depender de uno o pocos genes y tienen poca influencia del ambiente, mientras que los fenotipos continuos están influenciados tanto por múltiples genes como por el entorno. Los fenotipos que proporcionan ventajas adaptativas son seleccionados de manera natural, lo que, con el tiempo, lleva a cambios genéticos en la población. Así, la variabilidad genética y fenotípica es esencial para la supervivencia y la evolución de las especies frente a desafíos ambientales y presiones selectivas {\citep{Dercole}.}\\

            \normalsize{\citep{Mirrahimi} Desde la década de 1980, el término \textit{evolución adaptativa} se ha acuñado para describir los formalismos matemáticos que abordan la selección y evolución de una caracter\'iztica cuantitativa. Sin embargo en este trabajo se ocupa siguiente enfoque \citep{Dieckmann1996, metz96}: se estudia la evoluci\;on fenot\'ipica causada por mutaciones raras (en el sentido temporal) ignorando sexo y genes.}

%                \begin{itemize}
%                    \item {
%                    
%                        \normalsize{La multiplicación de la población.}
%
%                    }
%                    
%                    \item {
%                    
%                        \normalsize{La selección mediante competencia por los recursos disponibles.}
%                    }
%                    
%                    \item {
%                    
%                        \normalsize{Mutaciones.}
%
%                    }
%                \end{itemize}
                
            \normalsize{Modelos simples basados en estos principios pueden explicar como emergen rasgos mas aptos y, a su vez, como poblaciones caracterizadas por varios rasgos bien diferenciados pueden coexistir potencialmente. Las simulaciones numéricas pueden presentar la aparición de ciclos y la especiacion, esto debido a que los recursos limitados generan competencia; los individuos con características similares utilizan resucursos similares, dando asi, una competencia mayor entre ellos. La cuestión de comprender cómo, en un población de este tipo, una especie mutante puede invadir o no una población inicial. En modelos de población cerrada, las mutaciones forman parte de la dinámica y se toman en cuenta la heredad de los rasgos ligeramente diferente a los progenitores.}
        }

%-----------------------------------------------------------------
	\section{Conceptos Preliminares}{
		\subsection{\textit{Chemostat}}{
			\normalsize{El \textit{chemostat} o \textit{quimistato} es un dispositivo de laboratorio que está diseñado para el estudio del crecimiento de microorganismos en un ambiente controlado, en un medio líquido. Este dispositivo funciona de la siguiente forma: a un recipiente de vidrio cerrado, de entre 1 mL y unos pocos litros, se le suministra un medio fresco a través de una bomba de afluente. Para mantener un volumen constante, una segunda bomba extrae el líquido a la misma velocidad. Los microorganismos que han sido añadidos al recipiente únicamente pueden alimentarse del la bomba de afluente, y la tasa de crecimiento de esta está definida como la relación entre la tasa de afluente y el volumen del recipiente. Entre los sustratos y factores de crecimiento añadidos al medio, uno es el llamado sustrato de control, que limita el crecimiento \citep{chemostat}.}

            \begin{center}
                \includegraphics[scale=0.2]{Chemostat.jpg}
            \end{center}

            \normalsize{Algunas de las utilidades del quimiostato son que puede ser utilizado en cultivos puros para el estudio de la cinética del crecimiento microbiano, o para enfoques ómicos más detallados. También se puede utilizar para experimentos de competición \citep{chemostat}.}\\
            
            \normalsize{En estos experimentos se liberan en el recipiente dos o tres microorganismos diferentes, con nichos comparables en condiciones variables; con tasas de crecimiento altas o bajas; con concentraciones de oxígeno altas o bajas; distintos valores de pH o temperatura; con o sin factores de crecimiento, etc \citep{chemostat}.}

			\subsection{Ecuación de Hamilton-Jacobi.}
Desarrollamos la idea principal detr\'as de la ecuaci\'on con al que trabajaremo://
Sea una ecuaci\'on parcial diferencial no lineal \citep{evans} de la forma:
\begin{equation}\label{og}
	F(x;u,\nabla u)=0
\end{equation}
con $x\in\Omega\subseteq\Reals^n$, si renombramos $p\doteq\nabla u$ notaci\'on vectorial, asumimos que $F=F(x,u,p)$ es una funci\'on continua $F:\Reals^n$x$\Reals$x$\Reals^n \to \Reals$.\\
Dado $u(x)=\overline{u}(x)$ con $x\in\partial\Omega$ se construye una soluci\'on (al menos localmente, en la vecindad de la frontera) por el m\'etodo de caracter\'izticas.
Fijando un punto $x\in\partial\Omega$ consid\'erese una curva parametrizada por $t:x(t)$ con $x(0)=y$, y con:
\begin{equation*}
	\begin{split}
		u(t)&\doteq=u(x(t))\\
		p(t)&\doteq p(x(s))=\nabla u(x(t))
	\end{split}
\end{equation*}
Denotando por un punto la derivada con respecto a $t$ tenemos:
\begin{equation}\label{casi}
	\begin{split}
		\dot{u}&=\sum_i \frac{\partial u}{\partial x_i}\dot x_i=\sum_i p_i \dot x_i\\
		\dot p_j&=\sum_i \frac{\partial^2 u}{\partial x_i\partial x_j} \dot x_i
	\end{split}
\end{equation}

Diferenciando \eqref{og} con respecto a $x_j$:

\begin{equation*}
	\begin{split}
		\frac{\partial F}{\partial x_j}+\frac{\partial F}{\partial u}\frac{\partial u}{\partial x_j}+\sum_i \frac{\partial F}{\partial p_i}\frac{\partial^2 u}{\partial x_i\partial x_j}&=0\\
		-\frac{\partial F}{\partial x_j}-\frac{\partial F}{\partial u}p_j&=\sum_i \frac{\partial F}{\partial p_i}\frac{\partial^2 u}{\partial x_j\partial x_i}
	\end{split}
\end{equation*}

Al usar \eqref{casi} renombramos $\dot x_i=\frac{\partial F}{\partial p_i}$ obtenemos entonces:
\begin{equation*}
	\begin{split}
			\dot x_i&=\frac{\partial F}{\partial p_i}\\
			\dot{u}&=\sum_i p_i\frac{\partial F}{\partial p_i}\\
			\dot p_j&=-\frac{\partial F}{\partial x_j}-\frac{\partial F}{\partial u}p_j
	\end{split}
\end{equation*}
Que ahora, para un tipo espec\'ifico de problema supongamos que $F$ no depende explicitamente de $u$, entonces recuperamos las ecuaciones can\'onicas de Hamilton y podemos escribir $F\to\Ham$, y en notaci\'on vectorial.

\begin{equation*}
	\begin{split}
			\dot x&=\frac{\partial \Ham}{\partial p}\\
			\dot{u}&=p\cdot \frac{\partial \Ham}{\partial p}\\
			\dot p&=-\frac{\partial \Ham}{\partial x}
	\end{split}
\end{equation*}
con $x(0)=y$, $u(0)=u(y)$ y $p(0)=\nabla u(y)$.
De esta forma, podemos aplicar el formalismo de Hamilton-Jacobi a una variedad de problemas no necesariamente mec\'anicos, solamente pidiendo que la ecuaci\'on original cumpla $F=F(t;x,\nabla u)$. 


    
    }
%-----------------------------------------------------------------
	\section{Modelo}{
		%%%%========== TIPO DE DOCUMENTO ==========%%%%
\documentclass[letterpaper]{article}

%%%%========== IMPORTAMOS LAS PAQUETERIAS ==========%%%%
\usepackage[utf8]{inputenc}
\usepackage[spanish]{babel}
\usepackage[dvipsnames]{xcolor}
\usepackage{amssymb}
\usepackage{physics}
\usepackage{amsmath}
\usepackage{anysize}
\usepackage{multicol} 
\usepackage{graphicx}                      
\usepackage{blindtext}                                          
\usepackage{cancel}
\usepackage{tikz}

\usepackage{hyperref}
 \hypersetup{ colorlinks = true, linkcolor = black } %==== Hipervínculos ====%
\usepackage{geometry}
\newgeometry{ bottom = 2.54cm, top = 2.54cm, left = 2.54cm, right = 2.54cm } %==== Modificación de Márgenes ====%

\usepackage{fancyhdr}  
\pagestyle{fancy}
\fancyhf{}
\fancyhead[R]{Mecánica Clásica}
\fancyhead[L]{\thepage}
\renewcommand{\headrulewidth}{0.08pt} %==== Encabezados ====%

\renewcommand{\footrulewidth}{0.08pt} %==== Pie de Página ====%
\fancyfoot[L]{}
\fancyfoot[R]{\rightmark}

%\usepackage[pages=all]{background} 
%\backgroundsetup{ scale=1, color=black, opacity=0.15, angle=0,
 %   contents={ \includegraphics{logo_ifuap.png} }
%}

\setcounter{tocdepth}{3}

\newcommand{\Title}[1]{\begin{center} \LARGE{\textbf{\textit{#1}}} \end{center}}
\newcommand{\Abstract}[1]{\begin{abstract} \normalsize{#1} \end{abstract}}
\newcommand{\Theorem}[1]{\begin{center} \normalsize{\textit{#1}} \end{center}}

\newcommand{\blue}  { \color{blue} }
\newcommand{\red}   { \color{red} }
\newcommand{\green} { \color{OliveGreen} }
\newcommand{\orange}{ \color{orange} }

\newcommand{\deq}{:=}
\newcommand{\less}{<}
\newcommand{\grow}{>}

\newcommand{\Identity}{\mathbb{I}}
\newcommand{\Reals}{\mathbb{R}}
\newcommand{\Naturals}{\mathbb{N}}
\newcommand{\Integers}{\mathbb{Z}}
\newcommand{\Complex}{\mathbb{C}}
\newcommand{\Imaginaries}{\mathbb{I}}

\newcommand{\Lag}{\mathcal{L}}
\newcommand{\Ham}{\mathcal{H}}

\newcommand{\cbk}[1]{ \left( #1 \right) }
\newcommand{\sbk}[1]{ \left[ #1 \right] }
\newcommand{\kbk}[1]{ \left\{ #1 \right\} }
\newcommand{\tbk}[1]{ \langle #1 \rangle }

\newcommand{\Grad}{ \vec{\nabla} }
\newcommand{\Divg}{ \vec{\nabla} \cdot }
\newcommand{\Curl}{ \vec{\nabla} \times }
\newcommand{\Lapl}{ \nabla^{2} }

\newcommand{\rad}[1]{ \vec{r}_{#1} }

\newcommand{\define}[3]{ \left. #1 \right|_{#2}^{#3} }
\newcommand{\Sum}[3]{ \sum_{#1=#2}^{#3} }
\newcommand{\Tensor}[3]{ #1^{#2}_{#3} }

\newcommand{\Partial}[2]{ \frac{\partial#1}{\partial#2} }
\newcommand{\SecPartial}[3]{ \frac{\partial^{2}#1}{\partial#2\partial#3} }

\newcommand{\eqLagrange}[1]{ \frac{d}{dt}\cbk{\Partial[\dot{#1}]{\Lag}} - \Partial[#1]{\Lag} = 0 }

\newcommand{\eqHamilton}[2]{ \Partial[#1_{i}]{\Ham} = - \dot{#2}_{i}, ~ ~ \Partial[#2_{i}]{\Ham} = \dot{#1}_{i} }

\begin{document}
    \section{Modelo}{

        \subsection{Descripción del Entorno Biológico.}
        
        
        \normalsize{Consideremos un organismo que tiene que proporcionan energía (tales recursos se denominan sustituibles). Sean $S_1$ y $S_2$ las concentraciones de estos dos recursos contenidos en un quimiostato. Entonces el vector:}\\
        
        \begin{equation}
        I=\binom{S_{1}}{S_{2}},
        \end{equation}

        \normalsize{constituye la condición ambiental [7,8].}\\

        \normalsize{Los organismos pueden caracterizarse por diversos grados de consumir. Describimos estos rasgos por x y varían continuamente entre 0 y 1. Si el rasgo toma el valor 0, sólo se consume el recurso $S_2$ y cuando el rasgo toma el valor de 1 solo se consume el recurso $S_1$. El efecto general en el cual contribuyen los dos rasgos se describe a partir de los coeficientes $\eta(x)$  y $\xi(x)$, en el cual la cantidad promedio que consume un organismo con el rasgo x para los recursos 1 y 2 viene dado como:  $\eta(x)$ $S_1$ y $\xi(x)$ $S_2$ respectivamente.}\\

        \normalsize{En el caso de una población consumidora monomórfica, la dinámica ecológica está gobernada por el siguiente sistema de ecuaciones diferenciales:}\\

        \begin{equation}
        \begin{split}
             \frac{d S_1}{dx}&=S_{01}-S_1-\eta(x)S_1X\\  \frac{d S_2}{dx}&=S_{02}-S_2-\xi(x)S_2X\\ \frac{d S_1}{dx}&=-X+\eta(x)S_1X-\xi(x)S_2X 
        \end{split}
        \end{equation}

        \normalsize{Donde X representa la densidad de la población consumidora y $S_{01}$ es la concentración del recurso i en el medio de entrada.
        El sistema ecuaciones (2) tiene siempre que cumplir:}\\

        \begin{equation}
            \eta(x)S_1X-\xi(x)S_2X>1,
        \end{equation}

        \normalsize{y la tasa de crecimiento poblacional de los consumidores con el rasgo X, bajo condiciones ambientales constantes I, está dada por:}\\

        \begin{equation}
            r(x,I)=-1+\eta(x)S_1X+\xi(x)S_2X.
        \end{equation}

        \normalsize{Ahora, el análogo de (2) para la competencia de dos poblaciones consumidoras, una con el rasgo x y la otra con el rasgo y, está dado por el siguiente sistema:}\\

        \begin{equation}
        \begin{split}
            \frac{d S_1}{dt}&=S_{01}-S_1-\eta(x)S_1X_1-\eta(y)S_1X_2\\  \frac{d S_2}{dt}&=S_{02}-S_2-\xi(x)S_2X_1-\xi(y)S_2X_2\\ \frac{d X_1}{dx}&=-X_1+\eta(x)S_1X_1+\xi(x)S_2X_1\\
            \frac{d X_2}{dx}&=-X_2+\eta(y)S_1X_2+\xi(y)S_2X_1
        \end{split}
        \end{equation}

        \normalsize{En el estado estacionario, tanto como r(x,I) y r(y,I) son iguales a cero. Estas son dos ecuaciones lineales con dos incógnitas, $S_1$ y $S_2$. La solución se expresa como:}\\

        \begin{equation}
            \binom{S_1}{S_2}=\frac{1}{\eta(x)\xi(y)-\eta(y)\xi(x)}\binom{\xi(y)-\xi(x)}{\eta(y)-\eta(x)}
        \end{equation}

        
        \normalsize{A continuación, las dos relaciones de retroalimentación pueden utilizarse para deducir que las densidades en estado estacionario de las dos poblaciones consumidoras son:}\\

        \begin{equation}
            \binom{X_1}{X_2}=\frac{1}{\eta(x)\xi(y)-\eta(y)\xi(x)}\binom{\frac{\xi(y)S_{01}}{\xi(y)-\xi(x)}-\frac{\eta(y)S_{02}}{\eta(x)-\eta(y)}-\frac{\eta(y)-\xi(y)}{\eta(x)\xi(y)-\eta(y)\xi(x)}}{\frac{-\xi(x)S_{01}}{\xi(y)-\xi(x)}-\frac{\eta(x)S_{02}}{\eta(x)-\eta(y)}-\frac{\xi(x)-\eta(x)}{\eta(x)\xi(y)-\eta(y)\xi(x)}}
        \end{equation}

        \normalsize{De acuerdo con el Principio de Exclusión Competitiva, tres o más poblaciones consumidoras no pueden coexistir en estado estacionario utilizando solo dos recursos. De hecho, si r(x,I), r(y,I) y r(z,I) se igualan a cero, obtendremos tres ecuaciones lineales con solo dos incógnitas, lo que implica que, en general, no existe solución.}\\


        
        \subsection{Sistemas de Ecuaciones de Selección-Mutación y Paso al Límite para Mutaciones.}

        \normalsize{Si la reproducción no es completamente fiel, un consumidor con el rasgo yyy puede generar descendencia con el rasgo x. Sea K(x,y) la densidad de probabilidad correspondiente. En ese caso, se espera encontrar, con el tiempo, consumidores con todos los rasgos posibles. Sea n(t,.) la densidad de consumidores en el tiempo t. El sistema queda descrito por:}

        \begin{equation}
            \begin{split}
                \frac{d S_1(t,x)}{dt}&=S_{01}-S_1(t)+S_1(t)\int_{0}^{1}\eta(y)n(t,x)dx\\
               \frac{d S_1(t,x)}{dt}&=S_{02}-S_2(t)+S_2(t)\int_{0}^{1}\xi(x)n(t,x)dx\\
              \frac{d n(t,x)}{dt}&=-n(t,x)+\int_{0}^{1}K(x,y)[S_1(t)\eta(y)+S_2(t)\xi(x)]n(t,y)dy,
            \end{split}
        \end{equation}

        \normalsize{describe la interacción, a través de los recursos, de los diversos tipos de consumidores, así como el efecto de la mutación. Por lo tanto, se le denomina ecuación de selección-mutación (o sistema de ecuaciones). Por simplicidad, en la situación en la que la descendencia de un individuo con el rasgo x tiene una distribución de rasgos descrita por la densidad K(x,.).}\\

        \normalsize{Ahora, sea K(x,y) dependiente de un pequeño parámetro $\epsilon$; la idea es que las mutaciones son necesariamente pequeñas, lo cual incorporamos asumiendo que $K_\epsilon$ es insignificantemente pequeño para x fuera de un vecindario de radio $\epsilon$ alrededor de y [9].}\\

        \normalsize{Reescalamos el tiempo sustituyendo $\tau=\epsilon$t (este escalamiento ajusta la escala temporal de modo que, al hacer $\epsilon$ desaparecer, la escala de tiempo se adapte para observar el efecto de las mutaciones). Al escribir nuevamente t como $\tau$, ahora podemos reescribir la última ecuación de (4) como:}\\

        \begin{equation}
            \frac{\epsilon}{n(t,x)}\frac{d n(t,x)}{dt}=-1+\int_{0}^{1}K(x,y)[S_1(t)\eta(y)+S_2(t)\xi(x)]\frac{n(y,t)}{n(t,x)}dy.
        \end{equation}

        \normalsize{podemos luego realizar la siguiente transformación}\\

        \begin{equation}
            \varphi(t,x)=\epsilon ln[n(t,x)],
        \end{equation}

        \normalsize{mientras que el segundo término en el lado derecho puede escribirse como:}\\

        \begin{equation}
            \int_{0}^{1}K_{\epsilon}(x,y)[S_1(t)\eta(y)+S_2(t)\xi(x)]e^{\frac{\varphi(t,y)-\varphi(t,x)}{\epsilon}}dy
        \end{equation}

        \normalsize{Ahora supongamos que $K_{\epsilon}$ (x,y) es lo suficientemente pequeño para la variable y fuera de un vecindario de radio $\epsilon$ alrededor de x. Luego, realizamos el cambio de variable de integración y=x+$\epsilon$z y aproximamos:}\\

        \begin{equation}
            \frac{\varphi(t,y)-\varphi(t,x)}{\epsilon} \to \frac{d\varphi(t,x) }{dx}z
        \end{equation}

        \normalsize{Además, asumimos que la probabilidad de aparición de un nuevo rasgo como resultado de una mutación depende únicamente de la distancia al rasgo original. Por lo tanto, reemplazamos el kernel $K_{\epsilon}$ por un kernel de convolución $\widetilde{K}$ y aproximamos:}\\

        \begin{equation}
            K_{\epsilon}(x,y)dy\longrightarrow \widetilde{K}(z)dz
        \end{equation}

        \normalsize{Donde $\widetilde{K}$ es una función no negativa y par definida,cuya integral es igual a 1. Al tomar formalmente el límite cuando $\epsilon$ tiende a 0 en (9), obtenemos:}\\

        \begin{equation}
            \frac{d\varphi(t,x) }{dx}=-r(x,I)+[S_1(t)\eta(y)+S_2(t)\xi(x)]H(\frac{\partial \varphi(t,x) }{\partial x})
        \end{equation}
            

        \normalsize{donde r se define como en (4) y H se define por:}
        \begin{equation}
             H(p)=\int_{-\infty}^{\infty}\widetilde{K}(z)e^{-pz}dz-1
        \end{equation}

        \normalsize{Note que H(0)=0 y que, para una función par $\widetilde{K}$,se tiene $H'(0)>0$; por lo tanto,H es convexa. Llamamos a H el Hamiltoniano correspondiente a $\widetilde{K}$.}

\end{document}   
		\subsection{Descripción Superficial del Modelo Numérico.}
   \input{metodo_numerico}
	}
%-----------------------------------------------------------------
	\section{Conclusiones y Comentarios Finales}
	\clearpage
%---------------------------------------------------------------
	\bibliographystyle{plain}
	\bibliography{referencias.bib}
%------------------------------------------------
\end{document}
