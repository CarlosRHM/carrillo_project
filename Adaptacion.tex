\section{¿Qué es la Dinámica Adaptativa?}{

        \subsection{Adaptación y Evolución de los Sistemas Biológicos.}{

            \normalsize{En el estudio de los sistemas biológicos, los conceptos de adaptación y evolución son fundamentales para comprender cómo los organismos responden a su entorno y cómo estas respuestas moldean la diversidad de la vida en la Tierra. \citep{adaptacion} La adaptación se refiere al proceso mediante el cual los organismos desarrollan características que mejoran su capacidad para sobrevivir y reproducirse en condiciones ambientales específicas. Estas características pueden ser de naturaleza física, como la presencia de pelaje grueso en animales que habitan regiones frías; fisiológica, como la capacidad de las plantas desérticas para retener agua; o conductual, como la migración estacional de las aves. La adaptación resulta principalmente de la selección natural, un mecanismo que favorece aquellas variaciones genéticas que proporcionan ventajas competitivas en un entorno particular.}\\

            \normalsize{\citep {evolucion} Por otro lado, la evolución es un proceso a largo plazo que describe los cambios genéticos acumulativos en las poblaciones de organismos a lo largo de generaciones. Este fenómeno, impulsado por la selección natural, las mutaciones genéticas, la deriva genética y el flujo génico, es responsable de la aparición de nuevas especies y de la vasta diversidad de formas de vida observada hoy en día. La evolución permite entender cómo los organismos han cambiado con el tiempo para adaptarse a condiciones ambientales cambiantes, y cómo estos cambios han contribuido al desarrollo de estructuras, comportamientos y funciones altamente especializadas.}\\

            \normalsize{La relación entre adaptación y evolución es íntima y complementaria. Mientras que la adaptación representa la respuesta inmediata de un organismo o una población a presiones selectivas específicas, la evolución engloba los cambios genéticos a largo plazo que consolidan estas adaptaciones en las poblaciones. Por ejemplo, el cuello largo de las jirafas modernas es el resultado de un proceso evolutivo que tuvo su origen en la ventaja adaptativa de alcanzar las hojas altas de los árboles en entornos de escasez alimentaria.}\\

            \normalsize{\citep{Mirrahimi} Desde la década de 1980, el término \textit{evolución adaptativa} se ha acuñado para describir los formalismos matemáticos que abordan la selección y evolución de un rasgo en una población estructurada por un rasgo fenotípico continuo. Dichos modelos se basan en tres principios fundamentales que sustentan la evolución Darwineana:}
            
                \begin{itemize}
                    \item {
                    
                        \normalsize{La multiplicación de la población.}

                    }
                    
                    \item {
                    
                        \normalsize{La selección mediante competencia por los recursos disponibles.}
                    }
                    
                    \item {
                    
                        \normalsize{Mutaciones.}

                    }
                \end{itemize}
                
            \normalsize{Modelos simples basados en estos principios pueden explicar como emergen rasgos mas aptos y, a su vez, como poblaciones caracterizadas por varios rasgos bien diferenciados pueden coexistir potencialmente. Las simulaciones numéricas pueden presentar la aparición de ciclos y la especiacion, esto debido a que los recursos limitados generan competencia; los individuos con características similares utilizan resucursos similares, dando asi, una competencia mayor entre ellos. La cuestión de comprender cómo, en un población de este tipo, una especie mutante puede invadir o no una población inicial. En modelos de población cerrada, las mutaciones forman parte de la dinámica y se toman en cuenta la heredad de los rasgos ligeramente diferente a los progenitores.}
        }
        
        \subsection{Motivación para Introducir la Teoría de Hamilton-Jacobi.}

            \normalsize{\citep{Barles2006} Las ecuaiones de Hamilton-Jacobi son herramientas útiles para describir diversas asintósitcas singulares, es decir, situaciones en las que el comportamiento de un sistema físico o matemático cambia  de manera abrupta o presenta características extremas en ciertas condiciones, como cerca de puntos críticos, bordes o zonas donde se manifiestan discontinuidades. Estas presentan soluciones límite o aproximadas en condiciones extremas.
            }\\

            \normalsize{En los sistemas ecológicos y biológicos, la adaptación y evolución son procesos fundamentales que determinan la dinámica y supervivencia de las poblaciones. La capacidad de las especies para adaptarse a cambios en su entorno a través de la selección natural, la mutación y la competencia es un tema central en biología teórica. Para modelar estos procesos, las ecuaciones de Hamilton-Jacobi (H-J) han emergido como herramientas matemáticas poderosas para describir la evolución de poblaciones bajo distintos escenarios ecológicos.}\\

            \normalsize{\citep{Nicolas} La ecuación de Hamilton-Jacobi encuentra su aplicación en sistemas donde la dinámica de selección-mutación desempeña un papel clave. En este contexto, el cambio en el tiempo del valor de una cierta característica $\hat{s}$ en una población monomórfica, se da por medio de una ecuación de selección-mutación:}

            \begin{equation*}
                \frac{d \hat{s}}{d t}=\mu(\hat{s}) \frac{\sigma_0^2(\hat{s})}{2} n(\hat{s}) \partial_1 f(\hat{s}, \hat{s})
            \end{equation*}

            \normalsize{donde $\mu(s)$ es la probabilidad de que un nacimiento de un individuo con una característica $\hat{s}$ surja de una mutación; $\sigma_0^2(s)$ denota la varianza de distribución de una mutación $\hat{s'}$ proveniente de un individuo con característica $\hat{s}$; $\partial_1 f(\hat{s}, \hat{s}$ representa los parametros de intereacción entre individuos con característica $\hat{s}$ y $\hat{s'}$ dado por la natalidad y mortandad.}\\

            \normalsize{\citep{Calves} En el límite donde la tasa de mutación \( \mu \) es pequeña, las densidades de población tienden a concentrarse alrededor de valores particulares de \( x \), representando adaptaciones específicas. Este comportamiento puede analizarse mediante un ansatz de la forma:}

            \begin{equation*}
                u(x, t) \sim e^{\varphi(x, t)/\mu}
            \end{equation*}
                
            \normalsize{donde $\varphi(x,t)$ representa el frente de expansión. Sustituyendo este ansatz en la ecuación de selección-mutación y tomando el límite asintótico, se obtiene la ecuación de Hamilton-Jacobi para \( \varphi(x, t) \):}

            \begin{equation*}
                \partial_t \varphi(t, x)+H\left(I(t), x, d_x \varphi(t, x)\right)=0
            \end{equation*}
            
            }

