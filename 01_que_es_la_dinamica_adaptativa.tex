\section{¿Qué es la Dinámica Adaptativa?}

\subsection{Adaptación y Evolución de los Sistemas Biológicos.}

En el estudio de los sistemas biológicos, los conceptos de adaptación y evolución son fundamentales para comprender cómo los organismos responden a su entorno y cómo estas respuestas moldean la diversidad de la vida en la Tierra. \citep{adaptacion} La adaptación se refiere al proceso mediante el cual los organismos desarrollan características que mejoran su capacidad para sobrevivir y reproducirse en condiciones ambientales específicas. Estas características pueden ser de naturaleza física, como la presencia de pelaje grueso en animales que habitan regiones frías; fisiológica, como la capacidad de las plantas desérticas para retener agua; o conductual, como la migración estacional de las aves. La adaptación resulta principalmente de la selección natural, un mecanismo que favorece aquellas variaciones genéticas que proporcionan ventajas competitivas en un entorno particular.

\citep{evolucion} Por otro lado, la evolución es un proceso a largo plazo que describe los cambios genéticos acumulativos en las poblaciones de organismos a lo largo de generaciones. Este fenómeno, impulsado por la selección natural, las mutaciones genéticas, la deriva genética y el flujo génico, es responsable de la aparición de nuevas especies y de la vasta diversidad de formas de vida observada hoy en día. La evolución permite entender cómo los organismos han cambiado con el tiempo para adaptarse a condiciones ambientales cambiantes, y cómo estos cambios han contribuido al desarrollo de estructuras, comportamientos y funciones altamente especializadas.

La relación entre adaptación y evolución es íntima y complementaria. Mientras que la adaptación representa la respuesta inmediata de un organismo o una población a presiones selectivas específicas, la evolución engloba los cambios genéticos a largo plazo que consolidan estas adaptaciones en las poblaciones. Por ejemplo, el cuello largo de las jirafas modernas es el resultado de un proceso evolutivo que tuvo su origen en la ventaja adaptativa de alcanzar las hojas altas de los árboles en entornos de escasez alimentaria.

\citep{Mirrahimi2011} Desde la década de 1980, el término \textit{evolución adaptativa} se ha acuñado para describir los formalismos matemáticos que abordan la selección y evolución de un rasgo en una población estructurada por un rasgo fenotípico continuo. Dichos modelos se basan en tres principios fundamentales que sustentan la evolución Darwineana:

\begin{itemize}
	\item {

	      {La multiplicación de la población.}

	      }

	\item {

	      {La selección mediante competencia por los recursos disponibles.}
	      }

	\item {

	      {Mutaciones.}

	      }
\end{itemize}

Modelos simples basados en estos principios pueden explicar como emergen rasgos mas aptos y, a su vez, como poblaciones caracterizadas por varios rasgos bien diferenciados pueden coexistir potencialmente. Las simulaciones numéricas pueden presentar la aparición de ciclos y la especiacion, esto debido a que los recursos limitados generan competencia; los individuos con características similares utilizan recursos similares, dando asi, una competencia mayor entre ellos. La cuestión de comprender cómo, en un población de este tipo, una especie mutante puede invadir o no una población inicial. En modelos de población cerrada, las mutaciones forman parte de la dinámica y se toman en cuenta la heredad de los rasgos ligeramente diferente a los progenitores.

\subsection{Motivación para Introducir la Teoría de Hamilton-Jacobi.}

\citep{barles2007} Las ecuaciones de Hamilton-Jacobi son herramientas útiles para describir diversas asintóticas singulares, es decir, situaciones en las que el comportamiento de un sistema físico o matemático cambia  de manera abrupta o presenta características extremas en ciertas condiciones, como cerca de puntos críticos, bordes o zonas donde se manifiestan discontinuidades. Estas presentan soluciones límite o aproximadas en condiciones extremas.


En los sistemas ecológicos y biológicos, la adaptación y evolución son procesos fundamentales que determinan la dinámica y supervivencia de las poblaciones. La capacidad de las especies para adaptarse a cambios en su entorno a través de la selección natural, la mutación y la competencia es un tema central en biología teórica. Para modelar estos procesos, las ecuaciones de Hamilton-Jacobi (H-J) han emergido como herramientas matemáticas poderosas para describir la evolución de poblaciones bajo distintos escenarios ecológicos.

\citep{Mirrahimi2011}El ejemplo mas simple para un modelo matemático autocontenido para la evolución adaptativa es el quimiostato. Microorganismos caracterizados por un parámetro $x$ viven en un bao que contiene un nutriente continuamente renovado a una tasa $d>0$. Esta concentración se denota por $S(t)\geq0$ (para el sustrato), y el nutriente fresco se denota por $S(t)_{in}\geq0$. La densidad de población del microorganismo es $n(x,t)$, y la tasa de absorción para los individuos con rasgo $x$ es $\eta(x)>0$. Dando así las ecuaciones:

\begin{equation*}
	\left\{\begin{array}{l}
		\frac{d S(t)}{d t}=d\left(S_{i n}-S(t)\right)-S(t) \int_{-\infty}^{\infty} \eta(x) n(x, t) d x, \\
		\frac{\partial n(x, t)}{\partial t}=-d n(x, t)+(1-\mu) S(t) \eta(x) n(x, t)+\mu S(t) \int_{-\infty}^{\infty} M(y, x) \eta(y) n(y, t) d y
	\end{array}\right.
\end{equation*}

Los dos primeros principios mencionados en la teoría de Darwin están directamente incluidos en el modelo: el crecimiento de la población proviene de la ecuación de $n(x, t)$, y la competencia proviene de la cantidad limitada de nutrientes. Suponemos que inicialmente $S(0) \leq S_{i n}$, por lo que a lo largo de la dinámica $S(t) \leq S_{i n}$, ya que $S(t)$ disminuye si alcanza $S_{i n}$. El término (1$\mu) \eta(x) n(x, t)$ representa la tasa de nacimiento sin mutaciones. El parámetro $0<\mu<1$ representa la proporción de nacimientos que sufren mutaciones.

Las mutaciones están representadas por la probabilidad $M(y, x)$ de que un recién nacido tenga el rasgo $x$ cuando su progenitor tiene el rasgo $y$. Suponemos que $M(y, x) \geq 0$ y $\int_{-\infty}^{\infty} M(y, x) d x=1$

Este modelo se puede simplificar suponiendo que los nutrientes llegan rápidamente a un estado de equilibrio en comparación con la escala temporal de la evolución de la población. Esto se logra reemplazando la ecuación diferencial para $S(t)$ por:

\begin{equation*}
	S(t)=\frac{d S_{\text {in }}}{d+\int_{-\infty}^{\infty} \eta(x) n(x, t) d x} .
\end{equation*}

a su vez, reemplazamos el término de mutación por una ecuación de mutación-difusión:

\begin{equation*}
	\frac{\partial n(x, t)}{\partial t}=-d n(x, t)+S(t) \eta(x) n(x, t)+\lambda \Delta n(x, t) .
\end{equation*}

Con ello podemos construir un sistema de ecuaciones mas generales que pueden considerar tasas de crecimiento con/sin muerte $R(x,I(t)$:

\begin{equation*}
	\left\{\begin{array}{l}
		\frac{\partial n(x, t)}{\partial t}=n(x, t) R(x, I(t))+\lambda \Delta n(x, t), \quad x \in \mathbb{R}, t>0 \\
		I(t)=\int_{-\infty}^{\infty} \eta(x) n(x, t) d x
	\end{array}\right.
\end{equation*}

donde:

\begin{equation*}
	R(x,I)=-d+\frac{dS_{in}}{d+I}\eta(x)
\end{equation*}

\citep{Mirrahimi2011}Tales modelos parabólicos no pueden exhibir altas concentraciones mientras el coeficiente e difusión $\mu>0$ este fijo. Para ello, reescalamos el problema fijando un $\lambda=\varepsilon^2$. Asumiendo que la tasa de mutación es pequeña, considerando el paso al límite $\varepsilon \rightarrow 0$, lo cual lleva a la misma ecuación con $\lambda=0$, el modelo de selección. Esto es debió a que los efectos de mutación requieren tiempo muy prolongados para observarlos, por consiguiente, cambiamos $t$ por $t/\varepsilon$. Por consiguiente, las ecuaciones se transforman:

\begin{equation}
	\left\{\begin{array}{l}
		\frac{\partial n(x, t)_\varepsilon}{\partial t}=n(x, t)_\varepsilon R(x, I_\varepsilon(t))+ \varepsilon^2 \Delta n(x, t), \quad x \in \mathbb{R}, t>0 \\
		I_\varepsilon(t)=\int_{-\infty}^{\infty} \eta(x) n_\varepsilon(x, t) d x .
	\end{array}\right.
\end{equation}

Por otro lado, se puede demostrar bajo las condiciones:

\begin{equation*}
	\begin{cases}
		\sup_{x \in \mathbb{R}} R(x, I_M) = 0, \\
		\min_{x \in \mathbb{R}} R(x, I_m) = 0, \\
		\forall I \geq 0, \, R_I(x, I) < 0.
	\end{cases}
\end{equation*}

\begin{equation*}
	0< \eta_m \leq \eta(x) \leq \eta_M < \infty
\end{equation*}

Se puede demostrar que si $R$ es monótona en $x$ con datos iniciales "bien preparados", existen dos constantes $\rho_m >0$ y $\rho_M >0$, tales que:

\begin{equation*}
	\rho_m \leq \int_{-\infty}^{\infty} n_\varepsilon(x,t) \, dx \leq \rho_M
\end{equation*}

La demostración del Teorema anterior utiliza el método WKB, común en la propagación de frentes, aplicado en dinámica adaptativa. Este enfoque introduce una nueva ecuación de Hamilton-Jacobi debido a una restricción algebraica. La clave es la transformación de Hopf-Cole:

\begin{equation}
	u_\varepsilon = \varepsilon \ln(n_\varepsilon)
\end{equation}


Para esto, los datos iniciales deben estar ``bien preparados'', es decir, concentrados exponencialmente como \( u_0^\varepsilon = \varepsilon \ln(n_0^\varepsilon) \), con \( u_0^\varepsilon \) bien definido cuando \( \varepsilon \to 0 \). Esto lleva a la siguiente ecuación:

\begin{equation*}
	\frac{\partial u_\varepsilon}{\partial t} = R(x, I_\varepsilon(t)) +   \varepsilon \Delta u_\varepsilon + |\nabla u_\varepsilon|^2
\end{equation*}

Se puede demostrar que \( u_\varepsilon \) es uniformemente Lipschitz si \( u_0^\varepsilon \) también lo es, y que \( I_\varepsilon \) tiene variaciones acotadas. Al hacer \( \varepsilon \to 0 \), obtenemos la ecuación de Hamilton-Jacobi restringida:


\begin{equation*}
	\begin{cases}
		\frac{\partial u}{\partial t} = R(x, I(t)) + |\nabla u|^2 \\
		\max_{x \in \mathbb{R}} u(x, t) = 0, \quad \forall t > 0
	\end{cases}
\end{equation*}


La restricción \( \max_{x \in \mathbb{R}} u(x, t) = 0 \) proviene de la cota uniforme sobre la masa total establecida previamente. La solución \( u(x,t) \) se interpreta como solución de viscosidad.

\citep{Mirrahimi2011}En un quimiostato, la competencia entre especies es global y equitativa, pues todos se "alimentan" del único sustrato $S(t)$ introducido. Sin embargo, esto no siempre es el caso, en una situación más realista esta competencia no es equitativa, pues puede ser que un individuo con rasgos más cercanos tengan mayor habilidad de competencia. Para implementar esto, la dinámica población se modela con base a las ecuación tipo Lotka-Volterra:

\begin{equation*}
	\frac{\partial}{\partial t} n(x,t) - \lambda \frac{\partial^2}{\partial x^2} n(x,t) = n(x,t) \left( R(x) - (K * n)(x,t) \right), \quad t \geq 0, \, x \in \mathbb{R}
\end{equation*}

con la condición inicial $n(x,t=0)=n_0(x)$

Cuyos elementos tienen la siguiente interpretación:

\begin{itemize}
	\item \( n(x,t) \): densidad de población en la posición \( x \) y en el tiempo \( t \),
	\item \( R(x) > 0 \): tasa de crecimiento intrínseca de los individuos con el rasgo \( x \) (si están aislados y sin competencia),
	\item \( K \in L^\infty(\mathbb{R}) \): núcleo de competencia, una densidad de probabilidad tal que \( K \geq 0 \) y
	      \[
		      \int_{-\infty}^{\infty} K(z) \, dz = 1,
	      \]
	\item La convolución
	      \[
		      (K * n)(x,t) = \int_{-\infty}^{\infty} K(x - y) n(y,t) \, dy
	      \]
	      representa la competencia por recursos,
	\item \( \lambda \): tasa de mutación, asumida constante.
\end{itemize}